% Options for packages loaded elsewhere
\PassOptionsToPackage{unicode}{hyperref}
\PassOptionsToPackage{hyphens}{url}
%
\documentclass[
]{article}
\usepackage{lmodern}
\usepackage{amsmath}
\usepackage{ifxetex,ifluatex}
\ifnum 0\ifxetex 1\fi\ifluatex 1\fi=0 % if pdftex
  \usepackage[T1]{fontenc}
  \usepackage[utf8]{inputenc}
  \usepackage{textcomp} % provide euro and other symbols
  \usepackage{amssymb}
\else % if luatex or xetex
  \usepackage{unicode-math}
  \defaultfontfeatures{Scale=MatchLowercase}
  \defaultfontfeatures[\rmfamily]{Ligatures=TeX,Scale=1}
\fi
% Use upquote if available, for straight quotes in verbatim environments
\IfFileExists{upquote.sty}{\usepackage{upquote}}{}
\IfFileExists{microtype.sty}{% use microtype if available
  \usepackage[]{microtype}
  \UseMicrotypeSet[protrusion]{basicmath} % disable protrusion for tt fonts
}{}
\makeatletter
\@ifundefined{KOMAClassName}{% if non-KOMA class
  \IfFileExists{parskip.sty}{%
    \usepackage{parskip}
  }{% else
    \setlength{\parindent}{0pt}
    \setlength{\parskip}{6pt plus 2pt minus 1pt}}
}{% if KOMA class
  \KOMAoptions{parskip=half}}
\makeatother
\usepackage{xcolor}
\IfFileExists{xurl.sty}{\usepackage{xurl}}{} % add URL line breaks if available
\IfFileExists{bookmark.sty}{\usepackage{bookmark}}{\usepackage{hyperref}}
\hypersetup{
  pdftitle={Lab01\_Homework\_RMarkdown},
  pdfauthor={你是誰 R09342000 新聞所碩五},
  hidelinks,
  pdfcreator={LaTeX via pandoc}}
\urlstyle{same} % disable monospaced font for URLs
\usepackage[margin=1in]{geometry}
\usepackage{graphicx}
\makeatletter
\def\maxwidth{\ifdim\Gin@nat@width>\linewidth\linewidth\else\Gin@nat@width\fi}
\def\maxheight{\ifdim\Gin@nat@height>\textheight\textheight\else\Gin@nat@height\fi}
\makeatother
% Scale images if necessary, so that they will not overflow the page
% margins by default, and it is still possible to overwrite the defaults
% using explicit options in \includegraphics[width, height, ...]{}
\setkeys{Gin}{width=\maxwidth,height=\maxheight,keepaspectratio}
% Set default figure placement to htbp
\makeatletter
\def\fps@figure{htbp}
\makeatother
\setlength{\emergencystretch}{3em} % prevent overfull lines
\providecommand{\tightlist}{%
  \setlength{\itemsep}{0pt}\setlength{\parskip}{0pt}}
\setcounter{secnumdepth}{-\maxdimen} % remove section numbering
\ifluatex
  \usepackage{selnolig}  % disable illegal ligatures
\fi

\title{Lab01\_Homework\_RMarkdown}
\author{你是誰 R09342000 新聞所碩五}
\date{2021/02/22}

\begin{document}
\maketitle

{
\setcounter{tocdepth}{2}
\tableofcontents
}
\hypertarget{ux4f5cux696dux76eeux7684markdown-ux8a9eux6cd5ux5b78ux7fd2}{%
\subsection{作業目的:Markdown
語法學習}\label{ux4f5cux696dux76eeux7684markdown-ux8a9eux6cd5ux5b78ux7fd2}}

這份作業希望能夠讓你熟習於撰寫 Markdown 語法,希望你能夠成為 Markdown
界的霸主,聽起來很厲害。

請參考網路上的教學,你可以選擇仔細練習每個部份,或是有問題的時候再去查找上面的資源。緊接著就要開始寫第一份作業了,真是令人興奮!

你可以隨時點擊 RStudio 上方的 Knit 按鈕輸出檔案並且預覽,也可以利用
\href{https://jbt.github.io/markdown-editor}{Markdown
線上編輯器}看你的結果長得如何。

\hypertarget{ux4f5cux696d-rmarkdown-ux8a9eux6cd5ux7df4ux7fd2}{%
\subsection{作業: RMarkdown
語法練習}\label{ux4f5cux696d-rmarkdown-ux8a9eux6cd5ux7df4ux7fd2}}

滿分共 100 分。

\hypertarget{a.-ux8cc7ux6599ux65b0ux805eux6848ux4f8bux8209ux4f8b-60-ux5206}{%
\subsubsection{A. 資料新聞案例舉例 (60
分)}\label{a.-ux8cc7ux6599ux65b0ux805eux6848ux4f8bux8209ux4f8b-60-ux5206}}

請找一則資料新聞,並以 Markdown 語法介紹它。

介紹文必須包含底下者三種元素:字體加深, 清單(lists),
連結(links)。此外,請你再額外挑選至少兩種元素,譬如說字體變斜, 插入圖片,
加入引用, etc.

因為重點是使用 Markdown
語法,因此字數不用多,250字\textasciitilde400字即可,全文段落在
2段\textasciitilde4段間,請練習如何在 Markdown 中分行與分段!

請把結果貼在\texttt{\#\#\#\#\ 結果}下方。

\hypertarget{ux7d50ux679c}{%
\paragraph{結果}\label{ux7d50ux679c}}

\begin{figure}
\centering
\includegraphics{C:/Users/Hans/Desktop/P\#_title}
\caption{報導畫面}
\end{figure}

P\#新聞實驗室的\href{https://newmedia.pts.org.tw/subsidy/}{《選票變現金---200億補助款一次看》},是我覺得相當有趣的一則資料新聞。

根據報導內的說明,記者從中選會資料庫調閱出

\begin{itemize}
\tightlist
\item
  橫跨25年
\item
  共21場選舉
\item
  共6萬千餘筆資料
\end{itemize}

記者利用這些\textbf{候選人及政黨的補助款資料數據},向讀者說明\textbf{台灣選舉文化與政黨勢力消長之間的關係}。

我認為這則新聞作到的,並不只是彙整數據,而能帶領讀者從數字圖表呈現的趨勢中看見問題。

舉例來說,我們看見了民主進步黨所獲得的選舉補助款不斷攀升,已不再是最初那個資源匱乏、需被扶持的小黨。

同時我們也從\textbf{國、民兩大黨攫獲了這20多年來絕大多數款項的事實},發現選舉補助款的實施情況,與它\emph{「健全政黨政治」}的初衷,有些背道而馳。

我的童年

至於選舉補助款到底怎麼花,報導也邀請幾位候選人現身說法,節選如下:

\begin{quote}
我加入選舉,就是為了30元的補助款,我不諱言,這就是一個選舉花招。

by 屏東縣議員蔣月惠
\end{quote}

\begin{quote}
它是一體兩面的,否則未來選舉的情況就是,會參與選舉的人都是有本錢的人,很多人參政的機會就被排除掉了。

by 時任時代力量黨主席徐永明
\end{quote}

除了上述的內容,這篇專題也透過精美的網頁設計,邀請讀者在閱讀的過程參與互動,相當有趣。
唯一美中不足的是,這則新聞整理了過去,但卻無法藉以提出對未來的一些預測、假設或建議,讓報導的厚度有些不足。

\hypertarget{b.-r-code-chunk-ux7df4ux7fd2-40-ux5206}{%
\subsubsection{B. R code chunk 練習 (40
分)}\label{b.-r-code-chunk-ux7df4ux7fd2-40-ux5206}}

\hypertarget{b.01-20ux5206}{%
\paragraph{B.01 (20分)}\label{b.01-20ux5206}}

如先前所提,Rmarkdown 當中可以插入 code chunk,請參考
\href{https://rmarkdown.rstudio.com/lesson-3.html}{RStudio
的教學},在底下插入一段 code chunk,並在當中計算 \texttt{1+1}
的答案並將結果列印出來。

\hypertarget{b.02-20ux5206}{%
\paragraph{B.02 (20分)}\label{b.02-20ux5206}}

\begin{itemize}
\item
  請解釋 chunk header 中參數 include = FALSE
  會有什麼後果,並在底下加一段 include = FALSE 的 code
  chunk,一樣在裡面計算 \texttt{1+1} 的答案。
\item
  請解釋 chunk header 中參數 echo = FALSE 會有什麼後果,並在底下加一段
  echo = FALSE 的 code chunk,一樣在裡面計算 \texttt{1+1} 的答案。
\end{itemize}

\end{document}
